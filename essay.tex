% Please do not change the document class
\documentclass{scrartcl}

% Please do not change these packages
\usepackage[hidelinks]{hyperref}
\usepackage[none]{hyphenat}
\usepackage{setspace}
\doublespace

% You may add additional packages here
\usepackage{amsmath}

% Please include a clear, concise, and descriptive title
\title{Comp 210 Research Journal}

% Please do not change the subtitle
\subtitle{Comp 210 Assingment 3}

% Please put your student number in the author field
\author{1605629}

\begin{document}

\maketitle

\section{Making a VR/AR Game}
\subsection{Introduction}
Making VR games is becoming easier almost by the day as new software and hardware is released. However, with the added functionality comes a miriad of problems that must be overcome to make the VR gaming experience a fun one for it's users. This paper will address these issues and give options that can be undertaken as a designer to overcome them.

\subsection{Design}
The design of a game should be driven by the idea of creating interfaces within which the user can interact spontaneously without explainations \cite{spanogianopoulos2014human}. Environment Design and graphical quality are helping players to have  a better feel of the game \cite{mentzelopoulos2015hardware}. However, one of the key parts of the game is the game experience. This is driven by the design of the game. UI design is an extremely important thing when it comes to VR or AR games as it influences the player even more than in regular monitor based games. One flaw with AR games is that it can be effect by weather conditions or lighting as the camera quality will change. In fact Ventä-Olkkonen et al found that in a study of 35 people on the  Balance between Virtuality and Reality in Mobile Mixed Reality UI Design found that 28 people prefered a 3D model of a known location to show information to an AR simulations of a locations through a live camera feed \cite{venta2014investigating}. This shows that in most cases a simulated scene using the live camera is not the most suitable choice for an AR enviroment.

\subsection{Hardware Interfaces}
Although most modern VR kits use linked controllers for player interaction, one interesting piece of hardware that could be used is the the Microsoft Kinect sensor. This gives the uyser more freedom as it allows the player to interact with things in game without having to use controllers \cite {mentzelopoulos2015hardware} which will increase immersion. Another benefit of having a 'hands-free' tool like the Kinect is that it allows people with motor disability or lack of full control of their upper limbs to enjoy the gaming experience as they would struggle to operate hand held controllers \cite {oskoei2009application}.

\subsection{Useability}
Although a generic set of useability heuristics is the norm when it comes to designing games, emerging technologies have called for a more specific set of heuristi\cite{hvannberg2012exploitation}. Although a set of heuristics specific to VR have been set down by Sutcliffe and Gault \cite{sutcliffe2004heuristic}, it appears not many recent papers have referenced it during their work on VR heuristics which shows that often, designers much prefer to make their own heuristics than adopt someone else's \cite{hvannberg2012exploitation}.

\subsection{Homuncular Flexibility}
Another interesting phenomenon that can be taken into account is Homuncular Flixibility. This is the idea that users can be brought to identify with avatars whose bodies differ from their own \cite{won2015homuncular}.  This could be as simple as being a human of diffeerent age, race or gender, but can be as extreme as being a completely different species of animal. Another thing that can 'extend the users body' is the use of tools, even if the tool is as simple as a stick \cite{won2015homuncular}. Berti et al found that the use of tools remaps percieved far space as near space \cite{berti2000far}. Biocca also suggested that if technology changes the appearance and affordances of the body it also changes the self \cite{biocca1997cyborg}. It has been found that people have no problem controlling novel avatars if they are given sufficient feedback by the gamer \cite{won2015homuncular}.

\subsection{Multimodal Augmented Reality}
When approaching the development of an AR game there are several different methods one can follow. One useful but arguably underused method is that of multimodal AR. We must, however, start by defining the difference between Multimodal AR and Visual Perception AR. Whereas visual perception AR relies on superimposing virtual images onto real camera footagge, multimodal AR "allows complete sensory description of a users capabilities to compare systems" \cite{rosa2016re}. It could be argued that multimodal AR is slightly less accessable as it would require more hardware, but does allow for a more immersive experience as it stimulatesd more of the users senses \cite{rosa2016re}.

\subsection{Ethical Issues}
I will lightly touch on some of the ethical issues that making a VR or AR game can create. The previously mentioned phenomenon Homuncular Flexibility can cause some ethical issues. This is because it can change the user's sense of self. That is to say it can change how the users behaves, often without the user actually being aware of these changes \cite{madary2016real}. While having realistic movement for controls is a good way of increasing immersion, it can also have an increase in the players sense of presence which is known to cause aggressive thoughts and tendencies \cite{fumhe2015violent}/ A designer should take into account the affect this could have on their users while making a game. However, in most cases it could be deemed that the risks are not great enough to stop the development of a game.

\subsection{Conclusion}
It can be concluded that there are many more factors to take into account when designing a VR or AR game. They can often influence a user a lot more than a traditinal screen based game due to the increased immersion. There are many different modes of output for VR and AR that are viable, not just visual and audio outputs and much thought must be put into this so as to give the user the correct amount of feedback and not to cause the user discomfort by conveying these stimuli in an incorrect way. However, if these are achieved then a VR or AR game can be a highly effective means of entertaining its users, educating people or advertising. It is in a way, the next level of visual technology.



\bibliography{references}
\bibliographystyle{ieeetran}


\end{document}
